\documentclass[11pt,a4paper]{article}
\usepackage{amsmath}
\usepackage[utf8]{inputenc}
\usepackage{graphicx} 
\usepackage{listings}
\usepackage{concmath}
\usepackage[T1]{fontenc}
\usepackage[ngerman]{babel}
\usepackage{url}


\author{Moritz Basel}
\title{REST-Spezifikation zu "Interne Mitfahrgelegenheit"}
\date{\today{}}
\begin{document}
\normalfont
\maketitle{}
\tableofcontents{}
\section{Einführung}
\subsection{Zweck des Dokuments}
Dieses Dokument dient als Handbuch für Fronteddeveloper zur Software "Interne Mitfahrgelegenheit". Es wird versucht, die REST-API vollständig aufzuzählen und alle möglichen Anwendungsfälle zu erfassen.
Dennoch besteht die Möglichkeit, dass einzelne Spezialfälle (noch) nicht abgedeckt werden. Bei Auffinden undokumentierten Verhaltens, bei Featureanfragen und Sonstigen bitten wir um Rückmeldung unter <incomplete>.
\subsection{Terminologie}
\textbf{Endpoints} bezeichnen eine spezifische (z.b. \url{/api/check_api} oder gruppierte (z.B. \url{/api/users/<uid>}) URIs
\subsection{Allgemeines}
Alle Endpunkte lesen und schreiben ausschließlich JSON. Der Content-Type in jedem Http-Request-Header muss als 'application/json' vorliegen.

\section{User-Management}
\subsection{Überblick}
\begin{tabular}{|c|c|}
\hline
\textbf{POST} /api/users & Erstellt einen neuen Benutzer \\
\hline
\textbf{GET} /api/users/<UID> & User-Profil \\
\hline
\textbf{GET} /api/users/<Username> & User-Profil \\
\hline
\textbf{POST} /api/auth & Login-Token Erstellung \\
\hline
\end{tabular}
\subsection{\textbf{POST} /api/users}
Nötige JSON-Einträge:
\begin{itemize}
\item Username
\item Password
\item email
\end{itemize}

\subsection{\textbf{GET} /api/users/<UID>}
Agiert wie erwartet. Nur der Benutzer selbst sowie Globale Administratoren haben Zugriffsrechte.
\subsection{\textbf{POST} /api/auth}
Gibt den Login-Token im JSON-Eintrag 'access\_token' zurück. Dieser muss bei Zugriffen auf geschützte http-Endpunkte im Header-Feld 'Authorization' in allen Requests mitgegeben werden.
\section{Appointment-Funktionalitäten}
\subsection{Überblick}
\begin{tabular}{|c|c|}
\hline
\textbf{GET} /api/appointments/<appointmentID> & Daten zum gewähltem Appointment \\
\hline
\end{tabular}

\subsection{\textbf{GET} /api/appointments/<appointmentID>}
Not yet Implemented

\section{Entwicklungs-API}
\subsection{Überblick}

\begin{tabular}{|c|c|}
\hline
\textbf{DELETE} /api/dev/removeUser/<username> & Löscht bestimmten Nutzer vollständig \\
\hline
\textbf{GET} /api/dev/check\_token & Gibt die ID des momentanen Nutzers zurück \\
\hline
\textbf{GET} /api/dev/check\_api & Einfache Nachricht zum Testen der allgemeinen Erreichbarkeit \\
\hline
\end{tabular}

\subsection{\textbf{DELETE} /api/dev/removeUser/<username>}
Löscht den gewählten Nutzer vollständig. Kann nur vom Nutzer selbst und von globalen Administratoren durchgeführt werden.







\end{document}


